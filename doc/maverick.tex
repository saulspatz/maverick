\documentclass [12pt, letterpaper] {article}
\title {Maverick Solitaire}
\author {Saul Spatz}
\date {\today}
\usepackage {geometry}
\usepackage {fancyhdr}
\usepackage {amsmath , amsthm , amssymb}
\usepackage {graphicx}
\usepackage {hyperref}
\usepackage [english]{babel}
\usepackage [autostyle, english = american]{csquotes}
\MakeOuterQuote{"}
\begin {document}
\maketitle
\begin {abstract}
In Maverick solitaire, 25 cards are dealt at random, 
and the player attempts to partition them into five pat 
poker hands.  I have computed the exact probability of
winning, or rather, of the existence of a solution.  
The exact probability is ????.  In the remainder of this
article I describe how the background of the problem, and
how the computation was performed.   
\end {abstract}
\section{Introduction}
On January 19, 1958, when I was 11 years old, the popular television program 
\textit{Maverick} aired an episode titled 
\href{https://www.imdb.com/title/tt0644480/}{``Rope of Cards."} 
Bret Maverick, a gambler played by James Garner, bets that 
he can separate 25 randomly dealt cards from the ordinary 52-card 
French deck into five pat poker hands, where a pat hand is a 
straight, a flush, or a full house.  (Straight flushes and royal
flushes are considered special kinds of flushes in this game.)
He wins the bet, and later  states that the game can be won
"practically every time."  My personal recollection is that
he said "49 times out of 50," but I cannot find any support
for this, so I believe he must have said it in a later episode.

The story goes that the following day, novelty shops all over the
United States sold out of playing cards, as people tried the 
proposition for themselves.  In my home, card-playing was a popular
recreation, so we tried it out immediately after the show.  I have been
fascinated by this game ever since.  

Sometime time in the mid 1990's I did a statistical study of the game, and
found with a 99\% confidence level that the probability of winning 
(with perfect play,) is indeeed a bit more than 98\%.  I've always wanted 
to know the exact probability, and this article describes how I've computed it.

\section {The Rules}
Poker is played with a 52-card deck, with four suits (Clubs, Diamonds, Hearts, and Spades)
of 13 cards each.  The cards in each suit have ranks 
\begin{center}
         2, 3, 4, 5, 6, 7, 8, 9, 10, Jack, Queen, King ,Ace.
\end{center}
In what follows, Jack, Queen, King, and Ace will be abbrevaited as J, Q, K, A, respectively. 

A poker hand contains five cards, and the hands have certain values, of which only three concern us.
Five cards of consecutive ranks constitute a \textit{straight}.  While the Ace is normally the highest 
card of a suit, for this purpose it can also be considered the lowest card, so that 10 J Q K A and A 2 3 4 5
are straights.  Five cards of the same suit  constitute a \textit{flush}.  In poker a hand that is both a
flush and a straight, that is five consecutive ranks of the same suit is called a \textit{straight flush}, 
and is a very good hand indeed, but for our purposes it's just another flush.  The third type
of "pat hand" is called a \textit{full house}, and consists of three cards of one rank and two cards of another rank.
These are presumably called "pat hands" because in draw poker, a player holding one of these
hands must normally "stand pat", holding all five cards, for to draw a card is to destroy the 
value of the hand.

In Maverick poker, the player deals 25 cards at random, and attempts to separate them into five disjoint pat hands.
Of course, sometimes this is possible, but the solution is recondite, and the player may not find it.
We shall assume however, that the player plays perfectly, always solving the problem if a solution exists.  
Alternatively, we may just ask for the probability that a solution exists.

\section{Equivalence Classes}
There are \[\binom{52}{25}=477,551,179,875,952\] possible deals of 25 cards.  The suits don't really matter.  
If we permute the suits in a deal, the resulting deal has a solution if and only if the original deal did.  There are usually $4! =24$ ways to 
permute the suits, though sometimes there are fewer.  If two suits have exactly the same ranks, then interchanging those
two suits will not affect the deal at all.  In this case, there are only 12 equivalent deals, and if there are three identical suits, 
there are only four equivalent deal, since the only choice is which suit is the inequivalent one.

The plan is to generate one deal from each equivalence class, and determine whether it has a solution and how many deals it represents.
Then we need only consider about one tweny-fourth of the deals.  I did this by deciding that there would always be at least
as many Spades as Hearts, at least as many Hearts as Diamonds, and at least as many Diamonds as Clubs.  Furthermore,
if two of the suits had the same number of cards, then the suit that would normally be longer
will have the higher cards.  In poker, hand are compared lexicographically.  The highest cards in each hand are compared; the higher one belongs to the 
better had.  If the highest cards are the same, then the second-highest cards are compared, and so on.  Only
in the case where we have a tie all the way down the line do we have to adjust the number ofhands represented.

This is a substantial reduction, but we can do even better.  Besides the symmetry of suits, there is a symmetry of ranks.
Suppose we alter a hand by replacing every 2 by the K of the same suit, and every K by the 2 of the same suit; replacing
every 3 by the Q of the same suit, and every Q by the 4 of the same suit, and similarly exchanging 4 and J, 5 and 10, 6 and 9, and 7 and 8.
Aces are unchanged.  Clearly, the new deal has a solution if and only if the original deal has one.  The transformation takes flushes to 
flushes, full houses to full houses, and straights to straights (because the Ace can be high or low.)  
Now we need only consider about half the possible Spade suits.  We don't need to consider both a
suit and its "mirror image."  If the suit has a King but not a 2, we accept it; if it has a 2 but not a K, we reject it; if it has neitehr or both, 
we look to the 3 and Q to decide, and so on.  Again, it is possible that there are ties all the way down the line, and then we 
don't get a doubling.

Unfortunately, the rank and suit symmetries conflict, because transforming the ranks by reflection may change the ordering of the suits.  After relection
the suit that originally had the lowest low card will now have the highest high card, and the low cards probably had no role in determining
which of the original suits was better.  So, for example, if the Spades and Hearts have the same length, there is no way to combine the rule 
that the Spades are at least as good as the Hearts with the rank symmetry.  Still when all the suits have different lengths, we can use the rank symmetry.
This occurs in about 30\% of the deals, so we get a very worthwhile reduction.  Suit symmetries alone produced about 19.9 trillion equivalence 
classes.  Applying the rank symmetry reduces this number to 17,023,704,173,138.  Still a formidable number, but 460 trillion less than we started with.

\section{Exhaustive Search}

In mathematical terms, this is a "set exact cover" problem.  We are given a set (the 25 cards of the deal,) 
and a family of subsets of the set (all the pat hands that can be constructed from those cards).  The problem is 
to find a subfamily that will exactly "cover" the set, that is such that each element of the set belongs to exacly 
one member of the subfamily.  In Maverick solitaire, we must select five pat hands so that each card occurs 
in exactly one of the pat hands.

Problem like this are usually solve by "backtracking."  Pick one of the cards, and make a list of all the pat
nahs that include it.  Choose one of them, and the choose a card that is not covered by that hand.  Now make a
list of all the pat hands that include the second card, but no card in the first hand.  Choose one of of these,
and then choose another card that hasn't be covered yet, and make a list of all the pat hands that cover the th 







\end {document}
